% Einteilung des Papierformats, des Satzspiegels und einer Bindekorrektur
\documentclass[a4paper,DIV13,BCOR0cm,draft=TRUE]{scrartcl}

% Sprach-Einstellungen
% babel ermöglicht die Verwendung verschiedener Sprachen, z.B. in Zitaten.
% inputenc erlaubt die
% direkte Eingabe von Sonderzeichen, insbesondere deutscher Umlaute. Durch
% \usepackage[T1]{fontenc} werden von LaTeX Fonts in westeuropäischer
% Codierung verlangt.
\usepackage[english,ngerman]{babel}
\usepackage[utf8]{inputenc}
\usepackage[T1]{fontenc}

\usepackage{babelbib}
\usepackage{url}		% wird von babelbib gebraucht
\usepackage{apacite}

% Neudefinition des Bezeichners
\renewcommand*\refname{Quellen} % Standard: Literaturverzeichnis

\begin{document}

Beispiel für Referenzen zu verschiedenen Quellen aus verschiedenen bib-files.

\verb|\cite| einzeln: \cite{dummy2000} und \cite{DCRainmaker2018}

\verb|\cite| zusammen: \cite{dummy2000,DCRainmaker2018}

\verb|\citeNP| einzeln: \citeNP{dummy2000} und \citeNP{DCRainmaker2018}

\bibliographystyle{apacite}
\bibliography{bib1.bib,bib2.bib}

\end{document}
