% Datei: main.tex
% Begonnen: 03.02.2016 
% Letzte Änderung: 03.02.2016 
%
% Minimalbeispiel für Berechnung in einer Tabelle mit dem spreadtab-Paket.
% ToDo: Richtige Anordnung Dezimalkoma
% ToDo: Währungstrennzeichen nach Schweizer-Norm
% ToDo: Formeln kopieren mit \STcopy-Kommando (siehe spradtab-Manual)

\documentclass[a4paper]{scrartcl}

\usepackage[utf8]{inputenc}
\usepackage[T1]{fontenc}
\usepackage[ngerman]{babel}

\usepackage{spreadtab}

\begin{document}

\STautoround*{2}

\begin{spreadtab}{{tabular}{lrr|r}}
                    & @Anzahl & @EP & @Total \\
    \hline
    @Kongressgebühr &   1 & 490.0  & b2*c2 \\ 
    @Flug           &   2 & 316.15 & b3*c3 \\
    @Übernachtung   &   3 & 105.50 & b4*c4 \\
    @U-Bahn         &   8 & 2.20   & b5*c5 \\
    \hline
    @Gesamt         &     &        & d2+d3+d4+d5 \\
\end{spreadtab}

\end{document}

