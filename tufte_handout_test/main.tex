\documentclass[a4paper]{tufte-handout}

%\geometry{showframe}% for debugging purposes -- displays the margins

\usepackage[utf8]{inputenc}
%\usepackage{lmodern}	% Schrift "Latin Modern"
\usepackage[ngerman]{babel}
\usepackage{apacite}

\usepackage{amsmath}

% Set up the images/graphics package
\usepackage{graphicx}
\setkeys{Gin}{width=\linewidth,totalheight=\textheight,keepaspectratio}
\graphicspath{{graphics/}}

\title{Vorlage für Handout\thanks{Inspired by Edward~R. Tufte}}
\author[M. Strehler]{Marco Strehler}
\date{09.01.2012}

\usepackage{booktabs}

% The units package provides nice, non-stacked fractions and better spacing
% for units.
\usepackage{units}

% The fancyvrb package lets us customize the formatting of verbatim
% environments.  We use a slightly smaller font.
\usepackage{fancyvrb}
\fvset{fontsize=\normalsize}

% Small sections of multiple columns
\usepackage{multicol}



\begin{document}

\maketitle% this prints the handout title, author, and date

\begin{abstract}
\noindent Vorlage für ein Handout im Tufte-Style
\end{abstract}

%\printclassoptions

\section{Erstes Kapitel}\label{sec:erstes-kapitel}

\subsection{Unterkapitel}\label{sec:unterkapitel}

% let's start a new thought -- a new section
\newthought{Ein neuer Gedanke}, wird mit Kapitälchen eingeleitet.

\subsection{Sidenotes}\label{sec:sidenotes}
One of the most prominent and distinctive features of this style is the
extensive use of sidenotes.  There is a wide margin to provide ample room
for sidenotes and small figures.  Any \Verb|\footnote|s will automatically
be converted to sidenotes.\footnote{This is a sidenote that was entered
using the \texttt{\textbackslash footnote} command.}  If you'd like to place ancillary
information in the margin without the sidenote mark (the superscript
number), you can use the \Verb|\marginnote| command.\marginnote{This is a
margin note.  Notice that there isn't a number preceding the note, and
there is no number in the main text where this note was written.}

\cite{Smid2011}

\bibliographystyle{plainnat}
\bibliography{psych}

\end{document}
